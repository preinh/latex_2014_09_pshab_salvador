%\documentclass[a4paper,11pt]{meetingmins}
%\documentclass[a4paper,11pt,chair]{meetingmins}
\documentclass[a4paper,10pt,agenda]{meetingmins}

\usepackage[cm]{fullpage}
%\usepackage{showframe}

\usepackage[T1]{fontenc}
\usepackage[brazil]{babel}
\usepackage[utf8]{inputenc}
\usepackage[pdftex]{graphicx}


\setcommittee{PSHA\-B: Análise Probabilistica de Ameaça Sísmica no Brasil}
\setmembers{
\chair{M. Pirchiner (USP)},
\chair{S. Drouet (ON)},
\chair{M. Assumpção (USP)},

J.C. Dourado (UNESP),

J. Ferreira (UFRN),
J. Juliá (UFRN),
H. Bezerra (UFRN),

G.S. França (UnB),
L. Barros (UnB),
Kate Tomé (UnB),

Tania Bustamante (PUC-RJ),
Jackeline Castañeda (PUC-RJ),
Maria Cascão (PUC-RJ),
Celso Romanel (PUC-RJ),

L.C. Ribotta (IPT-SP),

Edgar Sanabria (B\&A),
Tatiana Romero (B\&A),

Sergio Hampshire (UFRJ/NBR),
Silvio Poli (UFRJ)

}
\setdate{24 de setembro de 2014}

\begin{document}
\maketitle

\section{Boas-Vindas}
	
	- PSHA-B (Grupo de Pesquisa em Ameaça e Engenharia Sísmica no Brasil)
	
	- Informes do último encontro (02/2014, RJ)
	
	- \textbf{Meta}: artigo discutindo a construção do modelo de ameaça sísmica
	para Brasil.
	

\section{Motivação}
	
	- Comissão ABNT para Estruturas Resistentes aos Sismos
	
	- Modelos sobre o Brasil: Giardini(GSHAP) e Petersen2010(preliminar)

		
\section{Aspectos Metodológicos (OpenQuake)}
		
	\subsection{i. fontes sismogênicas}
		\begin{itemize}[leftmargin=2.5cm]
		  	\itemsep0.0em
			\item diferentes abordagens (\textbf{sismicidade}, geologia e geodésia) \\
				para caracterização de fontes sismogênicas;
			\item remoção de agrupamentos;
			\item magnitude de completude;
			\item magnitude máxima e
			\item relações de recorrência (MFD)
		\end{itemize}
	
	\subsection{ii. magnitudes e rupturas associadas}

	\subsection{iii. relações de atenuação (GMPEs)}
	
	\subsection{iv. cálculo (propriamente dito) da ameaça}



\section{Modelos em construção no Brasil}
	
	- Brasil: Zonas. Dourado.

	- Brasil: Zonas. Assumpção.
	
	- Nordeste: Zonas. Joaquim / Jordi.
	
	- Sudeste: Zonas. Tania / Jackeline.
	
	- Zoneless / Smoothing. Marlon.
	


\section{Dados públicos disponíveis}
	
	- Catálogos: BSB2013.08 com magnitude Mw ?!
		
	- Falhas neotectônicas. (Hilário e/ou Ricomini)
		
	- Deformações crustais. SIRGAS.



\section{Árvore lógica para fontes sísmicas}
	
	- Usar outras abordagens (geológica, geodésica).
	
	- Pesos para a arvore lógica.


\section{Árvore lógica para GMPEs}
	
	- Dados disponíveis. Stéphane.

	- Critérios de seleção. Stéphane.


\section{Detalhamento dos Modelos (opcional)\\
	(à tarde em qualquer internet café ou restaurante)}
	
	- distribuição de profundidades,

	- direções preferenciais de ruptura, 

	- falhamentos.
	


\vspace{1em}

\nextmeeting{a definir.}
\end{document}
